\documentclass[a4paper,12pt]{article}
\usepackage[utf8]{inputenc}
\usepackage{amsfonts}
\usepackage{enumerate}
\usepackage{amsmath}
\usepackage{array}
\usepackage{xcolor}
\usepackage{graphicx}
\graphicspath{ {images/} }

\usepackage[top=1in,bottom=1in,left=0.5in,right=0.5in]{geometry}

\title{ 
        \includegraphics[scale=1.3]{logo-UNQ}\\ 
        \vspace{4mm}
        \huge{Parseo y generación de código  \\ 
        \vspace{4mm} 
        Trabajo Práctico - Analizador Léxico de Eiffel} }
\author{Jeremias Veliez - B. Emmanuel Pericon}
\date{Septiembre 26, 2025}

\renewcommand{\contentsname}{Indice}

\begin{document}

\maketitle

\tableofcontents

\newpage

\section{Introducción}

\vspace{4mm}
Implementar un analizador léxico para el lenguaje Eiffel, que permita reconocer
correctamente los tokens básicos de un programa escrito en dicho lenguaje. 
El analizador se debe construir utilizando las herramientas Flex (generador de analizadores
léxicos) y Bison (generador de analizadores sintácticos), aunque en esta etapa del trabajo
sólo se requerirá implementar la parte léxica.

\end{document}