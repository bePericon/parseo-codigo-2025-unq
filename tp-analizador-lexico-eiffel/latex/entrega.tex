\documentclass[a4paper,12pt]{article}
\usepackage[utf8]{inputenc}
\usepackage{amsfonts}
\usepackage{enumerate}
\usepackage{amsmath}
\usepackage{array}
\usepackage{xcolor}
\usepackage{graphicx}
\usepackage{listings}
\graphicspath{ {images/} }

\usepackage[top=1in,bottom=1in,left=0.5in,right=0.5in]{geometry}

\definecolor{codegreen}{rgb}{0,0.6,0}
\definecolor{codegray}{rgb}{0.5,0.5,0.5}
\definecolor{codepurple}{rgb}{0.58,0,0.82}
\definecolor{codered}{rgb}{0.6,0,0}
\definecolor{backcolour}{rgb}{0.95,0.95,0.92}

\lstdefinestyle{myeiffel}{
    backgroundcolor=\color{backcolour},
    commentstyle=\color{codegreen},
    keywordstyle=\color{magenta},
    numberstyle=\tiny\color{codegray},
    stringstyle=\color{codepurple},
    basicstyle=\ttfamily\footnotesize,
    breakatwhitespace=false,
    breaklines=true,
    captionpos=b,
    keepspaces=true,
    numbers=left,
    numbersep=5pt,
    showspaces=false,
    showstringspaces=false,
    showtabs=false,
    tabsize=2,
    emph={IDENTIFIER, RESERVED_TYPE, STRING, COMMENT, NUMBER},
    emphstyle=\color{magenta},
}

\lstdefinestyle{DOS}{
    backgroundcolor=\color{black},
    basicstyle=\scriptsize\color{white}\ttfamily,
    morekeywords={,tabular,toprule,midrule,bottomrule},
    otherkeywords={:, \$},
    keywordstyle=\color{green}, 
    emph={username, path},
    emphstyle=\color{green},
}

\title{
    \includegraphics[scale=1.3]{logo-UNQ}\\
    \vspace{4mm}
    \huge{Parseo y generación de código  \\
        \vspace{4mm}
        Trabajo Práctico - Analizador Léxico de Eiffel} }
\author{Jeremias Veliez - B. Emmanuel Pericon}
\date{Septiembre 26, 2025}

\renewcommand{\contentsname}{Indice}

\begin{document}

\maketitle

\tableofcontents

\newpage

\section{Introducción}

\vspace{4mm}
Implementar un analizador léxico para el lenguaje Eiffel, que permita reconocer
correctamente los tokens básicos de un programa escrito en dicho lenguaje.
El analizador se debe construir utilizando las herramientas Flex (generador de
analizadores
léxicos) y Bison (generador de analizadores sintácticos), aunque en esta etapa
del trabajo
sólo se requerirá implementar la parte léxica.

\vspace{6mm}
\section{Diseño y uso del analizador}
\vspace{4mm}

En este analizador tenemos dos archivos importantes, que son:

\begin{itemize}
    \item Por un lado el \textbf{eiffel.l}, que contiene la definición de
          las
          expresiones regulares de los tokens, directamente relacionado con
          \textbf{Flex}.
    \item Y por otro, el archivo \textbf{syntacticAnalyzer.y}, utilizado
          por
          \textbf{Bison} para coordinar la lectura de tokens y dar una
          salida entendible.
\end{itemize}

\noindent
Para esta entrega vamos a centrarnos en el primer archivo ya que el segundo
solo toma directamente los tokens y los devuelve.

\vspace{4mm}
\noindent
Necesitamos realizar una compilación de los archivos de la siguiente manera:

\begin{lstlisting}[style=DOS]
username:path$  flex eiffel.l
username:path$  bison -d syntacticAnalyzer.y
username:path$  gcc -o executable lex.yy.c syntacticAnalyzer.tab.c
\end{lstlisting}

\vspace{4mm}
\noindent
Por ejemplo, para el siguiente codigo en el archivo \textbf{tests/test.e}:

\begin{lstlisting}[language=Eiffel, style=myeiffel]
class HELLO_WORLD
create
    make
feature
    make
        do
            print ("Hola,utilizo Eiffel!%N")
        end
end
\end{lstlisting}

\vspace{4mm}
\noindent
A la hora de ejecutarse se realiza y ve de esta manera:

\begin{lstlisting}[style=DOS]
username:path$  ./executable  < tests/test.e 
TOKEN_CLASS                   -> class
TOKEN_IDENTIFIER              -> HELLO_WORLD
TOKEN_CREATE                  -> create
TOKEN_MAKE                    -> make
TOKEN_FEATURE                 -> feature
TOKEN_MAKE                    -> make
TOKEN_DO                      -> do
TOKEN_PRINT                   -> print
TOKEN_PARENTHESIS_OPEN        -> (
TOKEN_STRING                  -> "Hola,utilizo Eiffel!%N"
TOKEN_PARENTHESIS_CLOSE       -> )
TOKEN_END                     -> end
TOKEN_END                     -> end
\end{lstlisting}

\vspace{6mm}
\section{Expresiones regulares utilizadas}
\vspace{4mm}

Se trato de obtener la mayor cantidad de tokens posibles con las expresiones a continuación


\begin{lstlisting}[language=Eiffel, style=myeiffel]
"class"           { return CLASS; }
"inherit"         { return INHERIT; }
"create"          { return CREATE; }
"feature"         { return FEATURE; }
"main"            { return MAIN; }
"do"              { return DO; }
"from"            { return FROM; }
"until"           { return UNTIL; }
"loop"            { return LOOP; }
"end"             { return END; }
"if"              { return IF; }
"then"            { return THEN; }
"else"            { return ELSE; }
"make"            { return MAKE; }
"print"           { return PRINT; }
{RESERVED_TYPE}   { return RESERVED_TYPE; }
"+"               { return PLUS; }
"-"               { return MINUS; }
"*"               { return TIMES; }
"/"               { return DIVIDE; }
">"               { return GREATER; }
"<"               { return LESS; }
"="               { return EQUAL; }
"!="              { return NOT_EQUAL; }
">="              { return GREATER_EQUAL; }
"<="              { return LESS_EQUAL; }
"not"             { return NOT; }
"and"             { return AND; }
"or"              { return OR; }
"true"            { return TRUE; }
"false"           { return FALSE; }
":="              { return ASSIGN; }
":"               { return COLON; }
","               { return COMMA; }
"("               { return PARENTHESIS_OPEN; }
{STR}             { return STRING; }
")"               { return PARENTHESIS_CLOSE; }
{COMMENT}         { return COMMENT; }
{IDENTIFIER}      { return IDENTIFIER; }
{NUMBER}          { return NUMBER; }

[ \t\n\r]+        ;
.                 { return UNKNOWN; }
\end{lstlisting}

\newpage

\noindent
Algunos expresiones particulares fueron:

\begin{lstlisting}[style=myeiffel]
\\ Toda palabra con letras minusculas/mayusculas y/o guion bajo
IDENTIFIER [a-zA-Z][0-9_a-zA-Z]*  

\\ Uno o mas numeros
NUMBER [0-9]+

\\ Todo string, cualquier caracter que se encuentre entre comillas
STR \"[^\"]*\"

\\ Algunas palabras reservadas para tipos
RESERVED_TYPE (INTEGER|BOOLEAN|STRING)

\\ Los comentarios, estos comienzan con doble guion medio
COMMENT \-{2}.*

\\ Ignorar espacios y saltos de linea
[ \t\n\r]+ 

\\ El punto devuelve como UNKNOWN lo que no es reconocido por el resto de expresiones
.
\end{lstlisting}

\end{document}